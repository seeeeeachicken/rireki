\documentclass[a4j, 12pt]{jarticle}
%\usepackage{epsfig}
\usepackage{float}
\usepackage[dvipdfmx]{graphicx}
\restylefloat{figure}
\restylefloat{table}

\setlength{\topmargin}{-45pt}
\setlength{\oddsidemargin}{-1in}
\setlength{\evensidemargin}{0cm}
\setlength{\textheight}{29.6cm}
\setlength{\textwidth}{20.9cm}
\setlength{\headheight}{0cm}
\setlength{\headsep}{0cm}

\setlength{\unitlength}{1mm}

\begin{document}

\begin{picture}(209, 296)(0,-20)
	\put(100, 261){\makebox(100, 5)[l]{{ 2023年1月21日現在}}} %日付
	\put(146,227){\dashbox{0.5}(30,40)}
        \put(146, 227){\includegraphics[keepaspectratio, scale=0.1]{myface.jpg}}

	\put(23, 239){\framebox(116, 22)}
		\put(23, 256){\makebox(100, 5)[l]{{\scriptsize ふりがな}}}
		\put(23, 256){\makebox(100, 5){ふじた けんと}}
		\put(23, 256){\line(1,0){116}}		%ふりがなと氏名の間の線
		\put(23, 239){\makebox(100, 17)[tl]{{\scriptsize 氏名}}}
		\put(23, 239){\makebox(100, 17){藤田 賢斗}}
		\put(123, 239){\line(0,1){22}}		%性別・印との区切りの線
		\put(123, 256){\makebox(16, 5){性別}}
		\put(123, 239){\makebox(16, 17)[]{{男}}}
		
	\put(23, 222){\framebox(116, 17)}
		\put(23, 222){\makebox(70, 17){1995年12月30日生}}
		\put(23, 222){\makebox(70, 5)[r]{(満 28歳)}}
		\put(93, 222){\line(0,1){17}}
		\put(93, 222){\makebox(70, 17)[tl]{{\scriptsize 本籍}}}
		\put(93, 222){\makebox(70, 17)[l]{埼玉県}}
	
	\put(23, 202){\framebox(170, 20)}
		\put(23, 217){\makebox(132, 5)[l]{{\scriptsize ふりがな}}}
		\put(45, 217){\makebox(132, 5)[l]{{\scriptsize さいたまけんさいたましさくらくみなみもとじゅく1-14-30オアシスタイセイスリー202}}}
		\put(23, 217){\line(1,0){132}}		%ふりがなと住所の間の線
		\put(23, 202){\makebox(132, 15)[tl]{{\scriptsize 現住所}}}
		\put(23, 202){\makebox(132, 15)[l]{〒338-0831}}
        \put(23, 202){\makebox(132, 15)[b]{ \\埼玉県さいたま市桜区南元宿1-14-30 オアシスタイセイIII 202}}
		\put(155, 202){\line(0,1){20}}		%電話番号との区切りの線
		\put(155, 202){\makebox(38, 20)[tl]{{\scriptsize 携帯電話}}}
		\put(155, 212){\makebox(38, 5){{\scriptsize}(080)}}
		\put(155, 207){\dashbox{0.5}(38.1, 5){2213-1622}}
		\put(155, 202){\makebox(38, 5)[r]{{\scriptsize 方呼出}}}
		
	\put(23, 182){\framebox(170, 20)}
		\put(23, 197){\makebox(132, 5)[l]{{\scriptsize ふりがな}}}
		\put(45, 197){\makebox(132, 5)[l]{}}
		\put(23, 197){\line(1,0){132}}		%ふりがなと住所の間の線
		\put(23, 182){\makebox(132, 15)[tl]{{\scriptsize 連絡先}}}
		\put(23, 182){\makebox(132, 15)[tr]{{\scriptsize (現住所以外に連絡を希望する場合のみ記入)}}}
		\put(23, 182){\makebox(132, 15)[l]{メールアドレス: kento.2288@icloud.com}}
		\put(23, 182){\makebox(132, 15)[b]{}}
		\put(155, 182){\line(0,1){20}}		%電話番号との区切りの線
		\put(155, 182){\makebox(38, 20)[tl]{{\scriptsize 電話}}}
		\put(155, 192){\makebox(38, 5){{\scriptsize }()}}
		\put(155, 187){\dashbox{0.5}(38.1, 5){}}
		\put(155, 182){\makebox(38, 5)[r]{{\scriptsize 方呼出}}}


	\put(23, 174){\makebox(25, 5){年}}
	\put(48, 174){\makebox(13, 5){月}}
	\put(61, 174){\makebox(132, 5){学歴・職歴(各別にまとめて書く)}}

	\put(23, 164){\makebox(25, 10)}
	\put(48, 164){\makebox(13, 10)}
	\put(61, 164){\makebox(132, 10){学歴}}

	\put(23, 154){\makebox(25, 10){2014}}
	\put(48, 154){\makebox(13, 10){3}}
	\put(61, 154){\makebox(132, 10)[l]{私立 川越東高等学校 卒業}}

	\put(23, 144){\makebox(25, 10){2015}}
	\put(48, 144){\makebox(13, 10){3}}
	\put(61, 144){\makebox(132, 10)[l]{駿台予備学校池袋校 卒業}}

	\put(23, 134){\makebox(25, 10){2015}}
	\put(48, 134){\makebox(13, 10){4}}
	\put(61, 134){\makebox(132, 10)[l]{立教大学理学部数学科 入学}}

	\put(23, 124){\makebox(25, 10){2019}}
	\put(48, 124){\makebox(13, 10){3}}
	\put(61, 124){\makebox(132, 10)[l]{立教大学理学部数学科 入学}}
    
	\put(23, 114){\makebox(25, 10){2019}}
	\put(48, 114){\makebox(13, 10){4}}
	\put(61, 114){\makebox(132, 10)[l]{立教大学理学研究科数学専攻 修士課程 入学}}

	\put(23, 104){\makebox(25, 10){2021}}
	\put(48, 104){\makebox(13, 10){3}}
	\put(61, 104){\makebox(132, 10)[l]{立教大学理学研究科数学専攻 修士課程 卒業}}



	\put(23, 74){\makebox(25, 10){}}
	\put(48, 74){\makebox(13, 10){}}
	\put(61, 74){\makebox(132, 10){職歴}}

	\put(23, 64){\makebox(25, 10){2021}}
	\put(48, 64){\makebox(13, 10){4}}
	\put(61, 64){\makebox(132, 10)[l]{東日本電信電話株式会社 入社}}

	\put(23, 54){\makebox(25, 10){}}
	\put(48, 54){\makebox(13, 10){}}
	\put(61, 54){\makebox(132, 10)[l]{現在に至る}}


%-------------学歴・職歴欄罫線----------------------
	\put(23,4){\framebox(170,175)}	%外枠
	\multiput(23, 174)(0, -10){17}{\line(1,0){170}} %横罫線
	\put(48,4){\line(0,1){175}}
	\put(61,4){\line(0,1){175}}

\end{picture}

%==============================================================
\newpage
\begin{picture}(209, 296)(0,-20)

	\put(17, 256){\makebox(25, 5){年}}
	\put(42, 256){\makebox(13, 5){月}}
	\put(55, 256){\makebox(132, 5){学歴・職歴(各別にまとめて書く)}}

	\put(17, 246){\makebox(25, 10){}}
	\put(42, 246){\makebox(13, 10){}}
	\put(55, 246){\makebox(132, 10)[l]{}}

	\put(17, 236){\makebox(25, 10){}}
	\put(42, 236){\makebox(13, 10){}}
	\put(55, 236){\makebox(132, 10)[l]{}}

	\put(17, 226){\makebox(25, 10){}}
	\put(42, 226){\makebox(13, 10){}}
	\put(55, 226){\makebox(132, 10)[l]{}}

	\put(17, 216){\makebox(25, 10){}}
	\put(42, 216){\makebox(13, 10){}}
	\put(55, 216){\makebox(132, 10)[l]{}}

	\put(17, 206){\makebox(25, 10){}}
	\put(42, 206){\makebox(13, 10){}}
	\put(55, 206){\makebox(132, 10)[l]{}}

	\put(17, 196){\makebox(25, 10){}}
	\put(42, 196){\makebox(13, 10){}}
	\put(55, 196){\makebox(132, 10)[l]{}}

%-------------------------------------------------------
	\put(17, 191){\makebox(25, 5){年}}
	\put(42, 191){\makebox(13, 5){月}}
	\put(55, 191){\makebox(132, 5){免許・資格}}

	\put(17, 181){\makebox(25, 10){2015}}
	\put(42, 181){\makebox(13, 10){12}}
	\put(55, 181){\makebox(132, 10)[l]{普通自動車第一運転免許 取得}}

	\put(17, 171){\makebox(25, 10){2021}}
	\put(42, 171){\makebox(13, 10){5}}
	\put(55, 171){\makebox(132, 10)[l]{.com Master ADVANCE ★★ 合格}}

	\put(17, 161){\makebox(25, 10){2021}}
	\put(42, 161){\makebox(13, 10){8}}
	\put(55, 161){\makebox(132, 10)[l]{ITパスポート 合格}}

	\put(17, 151){\makebox(25, 10){2021}}
	\put(42, 151){\makebox(13, 10){10}}
	\put(55, 151){\makebox(132, 10)[l]{基本情報技術者 合格}}

	\put(17, 141){\makebox(25, 10){2022}}
	\put(42, 141){\makebox(13, 10){2}}
	\put(55, 141){\makebox(132, 10)[l]{CCNA 合格}}

	\put(17, 131){\makebox(25, 10){2023}}
	\put(42, 131){\makebox(13, 10){8}}
	\put(55, 131){\makebox(132, 10)[l]{統計検定準1級 優秀成績賞 合格}}

%-------------学歴・職歴欄&免許・資格欄罫線---------------
	\put(17, 131){\framebox(170, 130)}
	\multiput(17, 256)(0, -10){7}{\line(1,0){170}} %横罫線
	\multiput(17, 191)(0, -10){6}{\line(1,0){170}} %横罫線
	\put(42,131){\line(0,1){130}}
	\put(55,131){\line(0,1){130}}


%---------------------------------------------------------
	\put(17, 89){\framebox(170, 38)}
		\put(17, 92){\makebox(117, 38)[tl]{{\scriptsize 自己PR}}}
		\put(17, 89){\makebox(117, 38)[l]{
			\shortstack[l]{これまで主にプロジェクトマネジメントの業務経験を積んでき
            \\
            ましたが、今後はデータサイエンティスト業務にチャレンジし
            \\
            ていきたいと考えています。そのために統計学に関する勉強、
            \\
            情報収集を行っています。具体的には統計検定準1級の取得、
            \\
            書籍での勉強、オンライン学習サイトの活用、さらには関連
            \\
            分野であるAI分野の勉強も行っています。
            \\
            未経験の領域でも自ら学ぶ姿勢を持ち、早期にキャッチアップ
            \\
            していきたいと考えております。}
		}}
		\put(134, 89){\line(0,1){38}}		%縦線
		\put(134, 114){\makebox(53, 13)[tl]{{\scriptsize 通勤時間}}}
		\put(134, 114){\makebox(53, 13)[]{在宅勤務}}
		\put(134, 114){\line(1,0){53}}		%横線
		\put(134, 101){\makebox(53, 13)[tl]{{\scriptsize 扶養家族数(配偶者を除く)}}}
		\put(134, 101){\makebox(53, 13)[]{0人}}
		\put(134, 101){\line(1,0){53}}		%横線
			\put(134, 89){\makebox(26.5, 12)[tl]{{\scriptsize 配偶者}}}
			\put(134, 89){\makebox(26.5, 12){有}}		%有・無で記入
			\put(160.5, 89){\line(0,1){12}}		%縦線
			\put(160.5, 89){\makebox(26.5, 12)[tl]{{\scriptsize 配偶者の扶養義務}}}
			\put(160.5, 89){\makebox(26.5, 12){無}}		%有・無で記入
		


		
	% \put(17, 5){\framebox(170, 24)}
	% 	\put(17, 18){\dashbox(170.05, 11.05)}
	% 	\put(17, 18){\makebox(132, 11)[tl]{{\scriptsize 保護者(本人が未成年の場合のみ記入)}}}
	% 	\put(17, 18){\makebox(132, 11)[bl]{{\scriptsize ふりがな}}}
	% 	\put(37, 18){\makebox(112, 11)[bl]{}}	%ふりがなを記入
		
		

\end{picture}

\end{document}